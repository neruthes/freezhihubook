\begin{zhihuanswer}
\answerAuthor{轻狂124}\answerUrl{https://freezhihu.org/question/614052373/answer/64d6bce281fc7c55f9059ecd}短暂的波动而已。

未来经济萧条成为常态,被粉饰为``市场''手段的策略很容易就能``被证明''全部失灵,网左们自然会迎来他们全新的春天的。

真心劝说广义上的自由派千万别认为危机就一定能教育人,纵向看,饥荒一千次也没有诞生出新的东西;横向看,伊朗、俄罗斯都在经济受挫后投入了另一个反面。

不管后天塑造的比例有多少,人的本能是很难违逆的。

改开以来,本身就是在以生活水平的指数级增长来换取人对本能的背叛,那一旦低垂的果实全部摘完,承诺再也无法实现,要往哪走那是显而易见的。不仅是上层为了统治便利与合法性塑造得向特色左走,忍饥挨饿的群众也都会热情拥护多管一管。

说到底,没有思想上的解放,怎么走都是死路。

经济快速发展时期,各国都会有xx例外论的思潮,那越发展自然越要强调特殊性;反过来,经济萧条了,那显然更说明你那套行不通,得换个方向来试试。

类似的,直接经历危机,人们会下意识地寻找最直接最本能的答案:都存在哪些蛀虫啊敌人啊、国家要管管哪个方面哪个领域呀、要如何揪斗挖掘出潜在的坏分子让今晚吃顿好的啊。

可不经历危机,又不可能有足够的力量冲破组织的严密封控去形成新的共识。

所以,基本是死局。

类似的回答,我一年多前写过不止一个长答案(不知道在不在了),懒得赘述了。
\end{zhihuanswer}
