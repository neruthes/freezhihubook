\begin{zhihuanswer}
\answerAuthor{LaUnidadPopular}\answerUrl{https://freezhihu.org/question/614052373/answer/64d8e09a2f56f025f104e387}有啥反攻倒算,键政圈的事情,大部分就是图一乐,还``猖獗''和``反攻倒算''起来了。

真的做实践的←←壬,现在要么在搞劳工社会工作,要么在农村搞集体经济,这两年也有不少成果。

当然,根据网←的观点``实践的都进去了''。

不过话说回来,这个说法也有问题。第一,传销、地下教会、部分煌憨魔怔壬,不都是线下实践,他们也会进去啊,但是人家进去了照样还有在做的(从某种意义上来说,←←壬实践只要不魔怔和冲塔,其实绝大多数情况下是可以做下去的,主要是网←瞧不上这种实践,反而是这些土味老→有的是真,50W,其实更容易挨鉄拳),这个也不能是你就局限于键政的理由啊;第二,伊里奇都写过《左派幼稚病》,说不要无脑上,不要唯我独左,你们是一点没看进去啊,现在这种情况下搞你们自以为``大的要来''的实践本来就是送啊,我觉得现在持这种态度的``←壬''可能进去了对于社会主义事业反而是利大于弊的。

另外一点,就是绝大多数网←,实际上最多就是懂一点苏马的东西,这就导致了一个问题,苏马在汴京的时候对付土味老→和传统的工业资本主义那套管用,但是对付西方在冷战期间开发出来的专门对付苏马的新自由主义意识形态就有点力不从心了,对付新自由主义的恰恰是一些西马的东西,但是问题是第一西马那套比较专业化和晦涩难懂,所以说不便于普及,很多人读不懂;第二是西马的学者太多,很多人不知道看谁的。

网←因为自身内部有意见分歧的原因,现在是必然圈子化的,就是认同某一种观点的人逐渐形成互联网聊天群组,然后在群组里面互动,进而减少了在公共平台的发声。

因为在公共平台,你谈论某些话题必然就会引发质疑以及你是不是想让人去送人头的指责,因为大家都是隔着一道网线,你是谁?干什么的?大家都不清楚,毕竟又不是所有的人要和VMZ那样把身份证号都公开了。

不过,未来网←也不会消失的,关键是看他们这一群体会怎么发展,会不会因此有更多的人走向真的去做群众工作,去把←派的理论讲给人民群众,帮劳动者争取合法合理的权益,以及更进一步地去探索一种不同于新自由主义的发展模式这些还有待观察。

而且有的时候你也不得不说,官方意识形态本身就是有可能在不断的制造←壬的。

给大家讲个有意思的事情。

因为现在新版的高中政治课本必修一加了很多科学社会主义和国际共运史的内容,这些天和一些高中生聊天,发现他们讨论某些话题的时候经常蹦出来``意识形态''、``阶级斗争''之类带有左派色彩的词,一问都是课本上学的。
\end{zhihuanswer}
