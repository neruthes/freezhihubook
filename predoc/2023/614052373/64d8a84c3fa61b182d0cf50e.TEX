\begin{zhihuanswer}
\answerAuthor{知乎用户}\answerUrl{https://freezhihu.org/question/614052373/answer/64d8a84c3fa61b182d0cf50e}怎么说呢\ldots\ldots{}

顶多在你乎算是所谓``被反攻倒算''吧,,,

我看在别的各种平台,那反倒是比以往增多了不少,当然别的也是。

不过总体而言,二零年以来的思潮,总结起来就是:

激进化再激进化

某种程度上,网←是不够``激进化''的。

可以说从二零年开始进入了一个新局面:

一阶段:各种旗帜下的 改良 主张

二阶段:局部冲突和对抗思潮

三阶段:改良总破产与全面吃鸡

为什么说是一种新局面呢,因为过去几十年的 反塔
思潮和行动,因为当时还处于上升期等多种原因,本质上还是以一阶段居多,离真正的
激进化
还有一定距离。但短短几年真正的激进化思潮,不说别的,其苗头在短短三年内是飞速发展,可以说,这是前所未有的一种形势,是惊涛骇浪的前奏。

可以说,正在进行和在不远的将来愈发膨胀的,就是一种``改良总破产''的浪潮。

在改良总破产的浪潮持续行进的情况下,谁不够激进,谁还想着保,谁就会暂时性的占据劣势或者被所谓``反攻倒算''。当然,是暂时性的。因为持续激进化的最终结果必然会导向全面吃鸡,而谁是胜利者,那就看谁能在骇人的全面吃鸡中能够收拾巨浪过后满目疮痍的残局并最终取得胜利。

当然这个``被反攻倒算''的原因,倒不是跟``网←''自身有多大关系,更多的实际上是反塔思潮的附加品。其一,两三年前开始的这股浪潮中,很多``网←''本质是塔←,比如典中典乌有系或者思想火炬这种。另一方面,塔也利用网←为自己提供某种意识形态外衣来进行粉饰。事实上都不要说网←,京城中轴线上的那座纪念堂和承天门前挂着的画像就是最好的证明。而塔自身重新成为各方势力的靶子,连带着任何与塔相关的势力或者符号也会成为标靶,甚至比塔本身遭到的攻击更加猛烈。但是整个形势离所谓万劫不复倒还有一定距离。在这种情况下,各种形式的
旁敲侧击 就成为了更多被选择的方式。

说难听的,现在反←的可能很多人都没搞明白←到底啥主张,是个啥东西,只知道←跟塔有关,自己反塔所以也要反←而已,,,

当然塔嘛\ldots\ldots{}

您老人家都泥菩萨过江自身难保了还有心思在这不断削弱本阵营力量呢?您爱咋地就咋地,我倒是挺想看看您还能怎么作的。

还得提醒您老人家一句,塔最大的对手不是什么 康米目田民族派
这些,而是最庞大的日子人群体本身。您要真想着自己位子的安定的话,那建议狠狠对日子人全面出重拳捏。

最后从我这个前现代爱好者的角度来看嘛,网←不行和被``反攻倒算''的原因那就是:

你有点太温和了.jpg
\end{zhihuanswer}
