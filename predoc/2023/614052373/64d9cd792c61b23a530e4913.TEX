\begin{zhihuanswer}
\answerAuthor{月塔}\answerUrl{https://freezhihu.org/question/614052373/answer/64d9cd792c61b23a530e4913}中国所谓的``左派'',也就是大家常说的毛左派,其实就是建制派的边缘力量,现在经济形势不好,建制派自顾不暇,自然顾不上保护所谓的``左派''了。

关于这个问题,80年代已经讨论过了,当时王震副总理力主彻底清算所谓的左派,把他们当年干的好事全抖落出来,再建一个文革纪念馆,让子孙后代牢牢铭记。

但是诸位领导考虑再三,最终还是决定不清算左派,不但不能清算左派,还要尽力遮掩他们干的好事,具体的原因很简单,我之前已经说过了,这里不再赘述了。

此后的四十年里,毛左与建制派的关系十分微妙,一方面要抵制毛左的政治思想,毕竟列宁主义和市场经济的胜负早就分出来了,在世界其他地方根本不需要讨论。另一方面,建制派又要利用毛左的力量,制衡新兴的资产阶级和自由派,掩盖当年那些``必要的牺牲''、``无奈的抉择''、``时代的局限''、``伟大的探索''、``发展的阵痛''。

如果非要做个比喻的话,毛左与 建制派
的关系大概相当于俄罗斯政府与普京的关系, 久加诺夫
同志一方面维护着列宁,斯大林等前苏联领导人的名誉,宣传计划经济的优越性;另一方面又无条件的支持普京总统,创造各种神奇的理论证明普京总统的合法性,甚至普京总统痛批列宁,发动
俄乌战争 这件事,久加诺夫都要出来辩解一下:

当然,这个症状不止发生在俄共身上,中国所谓的左派也是如此,在乌有之乡的文章里,谁反对普京发动战争,谁就是汉奸、卖国贼,而中国左派青年们正摩拳擦掌,恨不得立刻赶赴前线,替普京大帝冲锋陷阵:

前两年由于种种原因,建制派气势如虹,各种反市场的操作层出不断, 网左
自然能够得势;但是自去年年中以来,建制派自身问题重重,自然顾不上维护毛左这些边缘力量了。
\end{zhihuanswer}
