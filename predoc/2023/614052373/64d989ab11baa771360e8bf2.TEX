\begin{zhihuanswer}
\answerAuthor{何裕徵}\answerUrl{https://freezhihu.org/question/614052373/answer/64d989ab11baa771360e8bf2}刚好刷到,贴来给这个问题下觉得``网左活该、不过昙花一现''的右友们提个醒:

你从什么时候开始感受到所谓的``资本的力量''? 1.7 万关注 · 1941 回答 问题

这么多一千个回答上万累积点赞可都是真实世界里的真实例子,如果你要是觉得都是当年(这可是都两年前的例子)``网左''们干的团建,那你大可以去下面发辟谣帖,看看关注这个问题的日子人们会怎么看待你。目前我可是一个勇敢的``辟谣''贴都没看见哦~

为什么这两年对左翼叙述反向解构的这么厉害?那明显是进一步的深化去前三十年意识形态遗产的行为,你去看新闻怎么看待招商引资就知道了,不意外的(现在意外那你早干什么去了)。热点这个东西是可以创造的,什么圈子都一样。而官僚只要出现,目前看来大部分媒体资本是没一个敢不速跪的(有估计也会没了)。他可以``假释''你一部分声音,扩大你一部分错误,那么必然也可以就此断章取义、放出反串和卧底。至于他的打手,哈哈,某些大v都不需要给钱,给点舆论风向马上就懂怎么去迎合了,毕竟流量上涨收益不是吗?赚小市民和
保守派
的钱也是不少的,甚至能比穷桂左翼学生钱多。他为何如此自信你肯定会内部瓦解?很简单,某些``左翼知识青年''越聪明就越愚笨,看似独立思考实际上全是他给你发的``禁书'',充满了蠢货能量的``禁书''------因为大众可能喜欢一个反派,但绝对不会喜欢蠢货。于是,便形成了完美的信息逻辑闭环游戏,从主角团到游客到反派,所有人都在他的运作下按部就班。

不过有个事情他无法做到彻底磨灭,那就是赤裸裸的现实。网左的出现不是什么组织转线上,而是线下幽灵的碎片在互联网角落里滋长出来的,所以这就是为什么右友们觉得``网左纱不尽''的原因。现实问题是无法用掩饰来磨灭的,一般来说只会有反面效果,比如鸭子和纪录片截图。你要是
丰年 我兴许给你付和两句,灾年老百姓能有心情吗?

所以,网络金鱼池里的我们其实无所谓自己帮那边的,毕竟电脑硬件不会因为
电脑病毒
烧起来。真正线上逻辑链的崩塌来源于现实对电脑的打击,也就是他已经无法靠一台电脑和现实人交差的时候。当那天到来,左派,或者说支持自由的,才能真正从这个
赛博 牢笼里逃出来。在此之前,我们不过是缸中之脑的篇面感知罢了。

那一天总会到来的。

\begin{center}\rule{0.5\linewidth}{0.5pt}\end{center}

8/14更新

一天不见,这个问题下居然来了个``顶天立地问心无愧''的``社民先生''。

从客观描述来看,这位先生选择用目田常用的``亚伯拉罕第四教''来表示对苏联的一种蔑称,觉得伊里奇篡夺了左派的冠名权,尤其是在苦难的赛里斯地区。同时,由于非法夺权,伊里奇在``通常框架下缺乏民主''。这点就足以引起重视了。我们知道,社民当中哪怕偏右派祖师爷考茨基也是仅仅在抨击苏联的建立对民主的背叛,并且弯弯绕绕地用各种经济、政治、社会角度说明``苏联必然不得好下场'',而不是觉得有苏联还不如那个拉着高尔察克和保皇派的``民主俄临时政府''继续生活在这个世界上。这位先生显然有自己的更先进的想法,他觉得苏联早期的发家不仅缺少合法性、缺少支持,而且本质上是用宗教洗脑突然从一个全俄吃鸡大赛小势力变成最终赢家的,并且和沙俄是一样的------借他的话进一步说,不如不革。

且不说伊里奇几句``all powers to the
soviet''怎么就变成万能洗脑咒了,很奇怪的是,就算布布这几句话真这么厉害,为什么伊里奇不派几个大牧首,比如圣朱加什维利,圣托洛茨基,圣季诺维也夫,圣布哈林这种,配合上大审判厅长圣捷尔任斯基,拉上大骑士团长圣图哈切夫斯基和圣伏罗希洛夫,为啥没有在苏波战争的时候一边战斗一边洗脑把波兰人也给感化成教徒呢?从《钢铁是怎样炼成》这种最基础``宗教读物''就可以窥见,布布对波兰的渗透和宣传是有在做的。如果第四教真的这么牛,为什么对波兰人就不成功呢?难道波兰有劳合乔治和克列孟梭自由骑士团铸就的自由圣光屏障吗?

出于不愿因给潜在反串乱扣帽子结果误伤社民正主的缘故,我便替各位广大社民朋友们去这位先生主页求证了一下,于是发现他是一个``社民右翼''。其实我本人出于严格区分的喜欢是把社民右翼单独拉出来称为``社会自由主义''的,毕竟如果``社民''谁都可以是那不就成没有任何归纳意义的大帐篷了吗?社民左还有卢森堡主义呢,人家那些老哥可没有认为可以和资接机合作。社民中间派一般是那些西方一般社民正当,通常不会在党章里去掉``公有制''/``接机斗争''这一条,毕竟去掉了那和自由派党或者说社自大类``社民右''也没啥区别了,只不过手段没那么激进而已。当然了,我也无意去除他左籍,毕竟``社民是什么''与其说是一个除籍问题不如说是一个历史遗留的分类问题,这里就暂且以这位先生为标准、把他看作是立场站左、手段往左右轴都在够的大帐篷吧。

当然了,绕这么大一圈还是没有解决我的疑问:他到底是在反串还是真的对他回答的最后一句话感到非常认同?于是,我翻到了他的这篇感想。

如果他这里没有任何反话存在的话,那我只能说\ldots 你似乎有点对自己太自信了。在你主动作为去追求现有框架下的民主时,就举几个比较严峻的例子,比如二月革命克伦斯基最后被两边嫌弃赶下台润美国、清末百日维新最后全部被慈禧清算、魏玛社民/自由派不愿联合搞公平竞争然后洗头老一键清理,哪一个是在合法程序下成功了的?如果你觉得形势没那么差,那你是觉得你现在去铂金找几个代表说服/拉人请愿游行/网上征集民意调查,如此合法合规在法律框架下进行,他和他的小伙伴们就可能被你说动、支持你明年开放自由选举吗?还是说,你也没试过,只是觉得``能成功''呢?

如果你觉得不行要被盯上的,那要么是你``左派社民''这套办法根本不在全世界通用适用(换而言之赛里斯就肯定不能用你的出路了),社民办法一样要被清算;要么你眼中的``左派社民正道''其实上是我眼中的``北欧社民''、即由自由派而不是社民的由慈善事业创造出来的幻想乌托邦天堂。如果是后者,那么这位``社民先生''确实不需要被清算,因为他靠着经济增长带来的自然生活水平提高就可以``实现梦想''了(至少对他本人来说是的),根本不需要什么激进手段,或者说,他靠着翼赞翼赞建制也可以过的不错,本身除了口头上支持没有对行动的要求,和我上面提到的建制目田思路差不多。

如果我上面这一串猜测、尤其是最后一个没猜错的话,那``社民''这个大帐篷确实需要在政治学上作进一步细分和划分了,不然总有一天会成为和``网左''一样可笑的滑稽帽子。
\end{zhihuanswer}
