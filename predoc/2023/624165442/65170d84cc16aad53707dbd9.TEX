\begin{zhihuanswer}
\answerAuthor{网线钳}美国有三种林肯。

一种是带领黑奴走向解放,为南方的黑奴逃脱奴隶主的剥削压迫的革命家林肯。

一种是带领联邦军一统美国,维护美国独立和领土主权完整的领袖林肯。

一种是大力发展美国工业水平,把美国从积贫积弱的蛮荒之地变成发达国家的建设者林肯。

在林肯亡故以后,南方人约翰逊继任总统,他对南方大力妥协,让革命家林肯显得不那么``契合时代要求'',于是革命家林肯死了,被人记住的是那位领袖林肯,``美国一点也不能少''。以及发展的工业家林肯,带领人民从艰难困苦中脱身。

于是在黑奴制度仍然隐形残留的今天,革命家林肯死去了,领袖林肯带领联邦军四渡密西西比河的故事仍然被美利坚联邦传唱,在南方邦联军猖獗的时候,那毅然决然的长征更是一次伟大的征程。而在英国带领加拿大二次入侵的时候,林肯对抗英战争的胜利乐观精神依然值得我们学习。

而建设者林肯的伟业也是如此,谁不记得美国内战结束草创后短短数年,钢产量就暴涨到原来的几十倍呢?谁不记得美国还没脱离战争的阴影的时候,如同幻想一般从生产线下来的大量战争兵器,是抗英战争想都不敢想的呢?

在如今的美国,林肯语录仍然有四卷流传于世,当今的美国孩子们都热爱那个林肯,热爱那个一统河山的英雄模样,那个运筹帷幄的建设者模样。

可谁还记得,那个穿着朴素的林肯,在华盛顿特区接见百万黑奴的模样呢?

缺失的林肯,还是林肯吗?

那些仍然在血汗工厂里面工作的黑奴,知道林肯也曾为自己而战,向南方奴隶主们发起坚决的斗争吗?

可总会有人记得的,美国的年轻小黑奴们,知道什么才是完整的林肯,什么才是活生生的人,他们知道,不砸碎那个虚伪的神像,活生生的人就不可能展示他的本来面貌。

可砸碎神像,何其难也,历史的风总是吹不到美国五十州大地。

黑奴们虽然地位卑微,可并不愚笨。他们善于使用借代来称呼林肯,如斯巴达克斯或者卡尔马克思,原因无他,林肯的名讳已经成为了名为反迪克西邦联实际上在做迪克西邦联过去做的事情的联邦政府的专属名词,哪怕这种称呼让外人听起来一头雾水,可这已经是最安全的办法了。

同理,对于弗雷德里克·道格拉斯被抓捕的行动内容,虽然联邦政府对此岁月史书颇多,可滑稽的点正在于拙劣的岁月史书---道格拉斯作为著名的民权领袖,联邦政府高层,难道买一点国宴喝剩的威士忌就是腐败吗?那么胡吃海塞法国海运牛排和高档红酒的约翰逊总统,又是什么呢?

可这些都不是能够明面上讨论的,自从约翰逊总统继任以后的《from USA United
lot of history
decison》颁布以来,讨论废奴运动已经成为了一项被禁止的行为,那么黑奴们也只能对着岁月史书取乐,这可能并非真的能改变什么,但是你总得给人最后一点,苦中作乐的东西吧。
\end{zhihuanswer}
