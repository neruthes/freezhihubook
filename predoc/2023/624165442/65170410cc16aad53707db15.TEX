\begin{zhihuanswer}
\answerAuthor{你我的噫鸨}\answerUrl{https://freezhihu.org/question/624165442/answer/65170410cc16aad53707db15}尽管很多人给这个梗上强度,

但至今(2023.9.29中秋节),这个梗仍旧只是一个梗。

它只试图完成解构,所以并不比秀才、大力王、春日影、银行不妙曲、原神启动有什么出众之处。

因为它们在解构了传统邻家帅大叔、传统日式爱抖露乐队、传统抽卡小人战斗手游的进程是成功的,且在同时的重新建构进行得也相当成功,这十分难得。

另一方面,即便不了解其原委,读者也能享受定型文的80\%乐子,远远超过``为什么不演奏春日影'',因为看懂这个梗的80\%,你还得补MyGO!!!!!前三集甚至前七集(看完第十集可以看懂90\%)

但它为什么能跨越数十年时空 成为
一个梗,那就是另一个话题了。这个话题不在这个问题的讨论范围之内,也不可能成为任何可以在知乎讨论的话题。

(顺便讲一句恐怖的,``大吼一声扑向''这六个字我已经在三次元的真实中小学生那里听到了几次了)

下面的半句话是对这个问题下的大部分答主说的:

缓则多是好事,会玩梗、有幽默和娱乐细胞,文学素养和艺术素养高的缓则多是大好事,如果还掺杂一些高科技人才,各大高等学府一起共襄盛举,那简直是国家之幸。

上一个这么热闹的时间地点是1750-1871,法兰西,我们今天用的米千克秒、科学框架和法律框架,甚至\_\_\_\_究竟应该是什么样的,都是那个时候法兰西人为主导敲定的
\end{zhihuanswer}
