\begin{zhihuanswer}
\answerAuthor{武田大膳大夫晴信}\answerUrl{https://freezhihu.org/question/624165442/answer/65170184cc16aad53707dadd}这没啥特别的, 定型文
也是个梗,梗这种东西都是一样的。比如说前年的时候,大家在套的模板是这样子的------

然后偶尔又有跟随时代的限定衍生版------

半年前呢?半年前大家玩梗的热情更针对那个什么我醉提酒入寒山,霜华满天:

顺便又弄出来了一堆衍生品

最近无非是又换了两个人继续耍而已。

总之,重要的是耍,至于谁成为耍的主角,反正耍梗的觉得无所谓。

一篇两篇的套皮小作文,于耍梗的人而言不过是一种发泄,也许是对现状的发泄,也许是跟风觉得好玩的发泄,也许是压力大了找点赛博啤酒瓶砸墙的发泄。

还有些人可能不是要发泄什么,只是活跃一下气氛,也玩一下流行。

这就是说,你不可能从各种梗里面找到什么世间真理的, 解构
的终点意味着精神空间的一片空白,没人指望在这里获得真理。

但是,这并不意味着这就是一种虚无主义。就像我以前说的那样,破圈是吸引圈外人感兴趣的第一步,耍完了梗,总会有人开始思考是怎么回事,然后开始自己读课外书。

读成什么样子,那是读书的人的事情。

读书本身,就是好事一桩。

等这个梗人们玩腻了,无非就是又产生一些新的梗。也许一部分人不喜欢一些梗,但是梗和跟风的人从来都是有的,这是客观现象,自古至今都不缺爱看热闹的人。

不过这些都没什么,梗是会过去的,而人们获得的东西就在那。

快乐,满足,思考,以及其他的你认为曾经有价值的东西。

定型文并不重要,它与其他的梗一样,什么也不代表,什么意义也没有。

重要的是,你想要什么,想怎么把这种东西化作对你有用的内容。

这就足够了。
\end{zhihuanswer}
