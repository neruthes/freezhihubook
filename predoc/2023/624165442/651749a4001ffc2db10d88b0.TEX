\begin{zhihuanswer}
\answerAuthor{大汉魏王曹孟德}\answerUrl{https://freezhihu.org/question/624165442/answer/651749a4001ffc2db10d88b0}一个滑梯,如果不能拿出史料来讨论,而是硬塞一些观点,甚至是一些和事实180度相反,抛开事实不谈的观点,那么,在赚钱容易,人人都忙着发财的时候,没人会去质疑这些观点。

当赚钱变难了,人们开始思考自己的劳动剩余被谁拿走了,这些观点就会被解构,因为这些观点是为劳保占有大量生产资料而服务的。

为什么生产力越发达,赚钱却越难了呢?就是因为某些劳保占有了大量生产资料,主要是房子股份理财产品等等。劳保通过占有这些,收走了大量的剩余。

许多东西都能被解构,但是到手的钱,是解构不了的。为了这个解构不了的东西,为了到手的钱多一些,消费多一些,劳动力再生产的成本低一些,大伙开始解构这些服务于劳保占有大量生产资料的观点。

由于和这些观点相关的事实的讨论受到极大的限制,于是人们会通过网红文的方式来解构。
\end{zhihuanswer}
