\begin{zhihuanswer}
\answerAuthor{庞安常}啥也没解构。叶永烈那篇文章有什么问题。难道王洪文被捕时乖乖束手就擒?那就请自己写一篇文章澄清一下吧。

王粉喜欢拿他买国宴剩下的茅台酒的事,说明他勤俭节约,艰苦朴素。其实爱占便宜和生活奢侈腐败并不矛盾啊。看另一个喜欢捡漏的大领导的业绩:

宋拓汉石经,据传是蔡文姬之父蔡邕书写,国内仅存3件,康生付了10元。

黄庭坚的《腊梅三咏》,真迹,康生付了5元。

宋画院仿赵干的《起蛟图》,康生只付了5毛。

不拿群众一针一线嘛,一毛不拔未免影响不好。占便宜顶多是人品问题,白嫖就是是纪律问题了。

买茅台的事应该出自这篇文章。
躲在毛泽东眼皮底下的腐败:王洪文爱喝茅台酒(2)------中新网
(chinanews.com.cn)

之后还有一大段呢。

当时的制度对官员公款吃喝限制得很严,但王洪文找到了变通的办法。徐景贤的回忆录《十年一梦》揭开了这个秘密。王洪文安排他在工总司的小兄弟马振龙当了上海市轻工局党委副书记兼革委副主任,直接掌握了试制试用产品的大权。王洪文去北京以后,马振龙就源源不断地给王洪文送试吃、试用的产品,从香烟、酒类到糖果、罐头、从手表、照相机、打火机到录音机、电视机,连王洪文设宴用的高级瓷器、玻璃器皿等,都由马振龙送去。高档食品、家用电器、日用百货、渔猎用具一应俱全的上海市轻工局,成了王洪文直接控制的物资供应站。而马振龙也越来越受到王洪文的重用和提拔,不但当了四届人大代表,而且经王洪文批准,连续到日本、阿尔巴尼亚等国访问。一番投桃报李的交换,使王洪文有了物质基础,过上了神仙般的生活。

其实这段的第一句就说明了为什么以小王的地位还要花钱买别人喝剩下的茅台。因为公款吃喝管得严。国宴那些茅台是要记账的,你直接拿还真不行。但是通过私人渠道他可让下属进贡了不少好东西哦。

徐景贤是上海帮的核心成员,应该也什么理由造谣自己老上司的私生活问题。如果王粉确实有证据证明小王两袖清风分文不取,就直接拿出来。搞些莫名其妙的``解构'',暗示别人污蔑你们的伟大领袖,其实自己才是只会脑补意淫的那个,那才是最搞笑的。
\end{zhihuanswer}
