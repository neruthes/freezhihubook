\begin{zhihuanswer}
\answerAuthor{zzzzzzzzzz}完整版:

豫州与解构 22 赞同 · 1 评论 文章

先说结论:

解构的就是塔。

本质上,就是豫州之后,
彻底走入了那种``秦失其鹿,天下共逐之''的局面。先前暗藏在地下的各门各派,在
掌控力式微的情况下一个个全从地下走出来。现时之局面,真可说得上是群雄逐鹿。

当然如果说这是某种意识形态占据上风的结果我其实不太认同,因为当今之局面主要是因为日子人从岁静逐渐走向激进,他们的目标有且只有一个那就是塔本身。如今在他们眼中,便是哪种学说有利于倒塔他们就倾向哪一边,于是我们可以看到形形色色的非塔非亲塔势力冒出头来,有时这个声量大有时那个声量大,因为本质上旗帜只是幌子。

换言之,我们可以借助前入关至圣先师嵩构先生的一句话来道明这一切:

``敏猪什么时候赢,只取决于圣朝什么时候把自己折腾完,跟敏猪没有任何关系''

类比,逻辑是一样的。

本质上,这一切的一切,只是在解构
,只不过有人更偏这个方向有人更偏那个方向。有人解构的时候更倾向于带上这派一块嘲讽,有人更倾向于带上那派而已,本质上,梗不过是工具。而工具这种东西,看的就是谁使用他,而不是工具本身带有什么什么特性。

或者说,建构的就是一种反塔逻辑。只不过这次与以往诸多情况不同的是,豫州之后,证明日子人是真的有一些武德在的。
\end{zhihuanswer}
