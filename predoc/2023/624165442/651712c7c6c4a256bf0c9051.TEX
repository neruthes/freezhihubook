\begin{zhihuanswer}
\answerAuthor{工农的心}李密在《陈情表》中指出,尊老敬老是西晋的国策,``伏惟圣朝以孝治天下。''

柳桢的笔记本
网友指出在晋朝这支有几千年传统文化中,``忠,在世家士族贵足知识分子眼中意味着个人充拜,意味着砖治毒才。

为了忠,有的人和父母、或子女绝裂,与世家贵足利益违背,在儒家看来简直是钢常颠倒,冠履倒置,灭觉壬性。

所以晋朝提倡以孝治天下。''

孝,就是宗族,就是人性,就是感恩,就是和谐。

先反对忠,于是两晋有儒家眼中的``思想接放,文化璀璨''。

故而,两晋以孝治天下。

鲁迅原文《魏晋风度及文章与药及酒之关系》``魏晋,是以孝治天下的,不孝,故不能不杀。为什么要以孝治天下呢?因为天位从禅让,即巧取豪夺而来,若主张以忠治天下,他们的立脚点便不稳,办事便棘手,立论也难了,所以一定要以孝治天下。''

------鲁迅《魏晋风度及文章与药及酒之关系》
\end{zhihuanswer}
