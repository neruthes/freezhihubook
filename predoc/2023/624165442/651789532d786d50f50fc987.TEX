\begin{zhihuanswer}
\answerAuthor{kezzy}\answerUrl{https://freezhihu.org/question/624165442/answer/651789532d786d50f50fc987}这种文就是网络民粹边缘人的流氓主义的话术。。。用流氓的方式,解构正统话语。因为这是最方便的。

这都是玩了一百多年的东西了,认真说都是左圈玩剩下的,右圈其实倒比较拥护政府。。。以前,潘汉年在创造社就玩这一套,玩新流氓主义,
阴阳怪气,冷嘲热讽,脏话连篇,解构政府合法性。这种方式,特别适合,对方话语强势,你弱小的时候。对方虽然强势,但搁不住创造社玩我是流氓我怕谁这一套。

这种玩意儿往往会流行。鲁迅当年就发现,自己认真写的批判传统文化的刊物,根本卖不过这种流氓文。

后来连创造社老大郭沫若都看不下去了,觉得这样也太低俗了些。郭沫若虽然也是流氓文风,但他毕竟是海归精英,跟潘汉年这种中学毕业的小流氓不一样。

再往前,魏晋名士就这样。你司马家族不是以孝治天下吗?咱哥们就故意耍流氓。随地大小便啊,裸奔啊。解构你那套正统。

像小王定型文这种玩意儿,认真说,除了发泄情绪,啥意义也没有。纯纯就是耍流氓行为。很多人觉得这是自由民主,殊不知这其实是网络民粹。是跟自由主义截然相反的。自由主义讲究的是精英主义。
\end{zhihuanswer}
